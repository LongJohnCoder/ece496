\section*{Group Highlights}

In order to build a working \emph{Overlay FPGA}, we produced the following:
\begin{itemlist}
\item Verilog for the individual FPGA building-blocks: the logic cell, logic block, connection block, and the switch block. (Keyi)
\item Verilog to connect the building block components together in a grid and feed the programming signals through them. (Keyi)
\item Verilog circuit to interface with the UART to enable serial programming of the Overlay FPGA. (Neil)
\item Scripts to run the third-party tools which will process an input circuit (from the user) and produce a valid placement and routing. Some adjustments need to be made so the tools can read each others' outputs. (Neil)
\item Software to convert the placement and routing into a programmable bitstream. (Neil)
\end{itemlist}

Our Overlay FPGA circuit meets the functional requirements and objectives set out in our project proposal.

We found that overhead of implementing and FPGA on an FPGA was higher than we expected because some resources we used extensively were limited.
Due to this limitation, we reduced our target size constraint for the Overlay FPGA from 3000 logic cells down to 100 logic cells. % FIXME: 100?
Although this sounds much lower, it is still large enough to implement working overlay designs.

The size restriction limits the size of circuits our \overlay can support, so we can no longer accommodate the MCNC benchmark circuits we originally planned to use.
We developed our own simple \emph{unit-test} circuits which each test very specific functionality.
These new test circuits allowed us to incrementally track down basic issues with our bitstream.

