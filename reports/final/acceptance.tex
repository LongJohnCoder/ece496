\section{Validation and Accptance Tests from Proposal} % from project proposal

\subsection{Functional validation}

To validate the functional requirements, we will:
\begin{enumeration}
	\item configure the Overlay FGPA and program it onto the physical FPGA,
	\item select and use a benchmark circuit commonly used to test VPR,
	\item configure VPR to match our architecture and dimensions,
	\item place and route the benchmark circuit with VPR,
	\item convert the VPR output into a bitstream for the overlay FPGA,
	\item load the bitstream onto the overlay FPGA, then 
	\item test the functionality of the benchmark circuit running on the overlay FPGA.
\end{enumeration}

The exact verification process for inputs and outputs will depend on the benchmark circuit's intended function.
For the simple test circuits we will test initially, we will set inputs using hardware switches, and observe outputs on LEDs.

Intermediate outputs can also be verified throughout the development process.
We can dump out the bitstream to a text file to verify that it matches the placement and routing in VPR.
We can test the hardware component independently by writing a bitstream by hand as well.


\subsection{Size and overhead validation}

To ensure that the overhead is low enough that the overlay FPGA can fit useful circuits, we will test it using the \emph{``Golden 20''} MCNC benchmark circuits.
For each circuit, we will:
\begin{enumeration}
	\item run synthesis and technology mapping using the ABC synthesis tool,
	\item run placement and routing using VPR configured, and
	\item confirm that VPR can place and route the benchmark circuit using the number and arrangement of logic blocks that we can fit.
\end{enumeration}

