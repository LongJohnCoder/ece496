\section{Summary} % and Conclusions

Overall, the project succeeded in what we set out to do.
We met all of our final goals and requirements, producing a flexible overlay FPGA design and tool flow that is capable of implementing a user-specified verilog circuit on the overlay.
We were able to produce a functioning buffer and adder on an 8x8 overlay using our custom tool flow, demonstrating the functionality of all parts of our project (overlay circuit, bitstream generation, bitstream injection).

We believe that our design idea has the potential to become a useful tool for FPGA researchers to test custom architectures or CAD algorithms on real hardware.
The current design is a successful proof-of-concept; it may be useful to some researchers, but others may require a lower overhead.
Some overhead reduction can be achieved by optimizing the structure of the \overlay, by using Clos networks for example.

Modern FPGAs include a lot of fixed-function blocks (\emph{hard macros}) that implement common functions such as multipliers and memories in order to reduce the overhead of using an FPGA for circuits containing many instances of these elements.
A possible extension to our project would be to encorporate \emph{hard macros} available on the Virtex 5 in our overlay circuit.

With our \overlay design, a researcher could easily instantiate the overlay in an FPGA-based embedded system with a \emph{soft microprocessor} such as Xilinx's MicroBlaze.
This would enable research on using re-programmable hardware in embedded systems.

% Did you meet your project goals and requirements, as demonstrated through your validation and acceptance tests?
% To what extent does your testing and verification work prove out your final design?
%  Were your design ideas proven out? If not, explain why.
% What are the key conclusions to be drawn from your project?
% Where is the kind of work you did in your project used or potentially useful in state-of-the-art, industry, academe, or society?


